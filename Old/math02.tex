\documentclass{article}
\usepackage{amsfonts,amssymb}
\usepackage[margin=1in]{geometry}
\usepackage{fancyvrb}

\newcommand{\qed}{\hfill\ensuremath{\blacksquare}}
\title{CSCI 480, Winter 2017\\Math Exercises \# 2}
\author{Andrew Nguyen}
\date{Due date:  Wednesday, February 1, midnight.}

\begin{document}

\maketitle


\noindent{\large
  {\bf Turn in both the tex and pdf files (not zipped):  } {\tt math02.tex} and {\tt
  math02.pdf}.  }

\begin{description}
\item[Exercises for Chapter 4]
Use the method of direct proof to prove the following statements.
\item[16.] If two integers have the same parity, then their sum is even.

Proposition: The sum of two integers with the same parity is an even integer.

\textit{Proof.} Suppose that we have two integers, $a$ and $b$, that have the same parity.

The set of all even numbers is defined as $\{2n:n\in\mathbb{Z}\}$.

The set of all odd odd numbers is defined as $\{2n+1:n\in\mathbb{Z}\}$.

Thus, the set of the sum of $a$ and $b$ is either 

$(2n+2n) = (4n)$, if $a$ and $b$ are even,

or, 

$((2n+1)+(2n+1)) = (4n+2)$, if $a$ and $b$ are odd.

For any integer, $\{n\in\mathbb{Z}\}$, whether the summed number falls into
$(4n)$ or $(4n+2)$, the integer will be even.

Therefore, the sum of any two integers with the same parity is an even integer.

\item[Exercises for Chapter 5]
Use the method of contrapositive proof to prove the following statements.
\item[12.]  Suppose $a\in\mathbb{Z}$.  If $a^2$ is not divisible by 4,
  then $a$ is odd.
  
  Proposition: Suppose $a\in\mathbb{Z}$. If $a^2$ is not divisible by 4,
  then $a$ is odd.
  
  \textit{Proof.} Suppose $a$ is even.
  
  Since $a$ is even, the set of all $a$ in $\mathbb{Z}$ is defined as:
    $\{2n:n\in\mathbb{Z}\}$
  
  Accordingly, the set of all $a^2$ in $\mathbb{Z}$ must be: $\{4n^2:n\in\mathbb{Z}\}$;
  this is the set of all values of $a^2$ where $a$ is an even integer.
  
  Since $4n^2$ is a multiple of 4, all the values of $a^2$, where $a$ is even, must also be divisible by 4.
  
  Therefore, for all values of $a^2$ that are not divisible by 4, $a$ must be odd.

\item[Exercises for Chapter 6]
Use the method of proof by contradiction to prove the following statements.
\item[18.] Suppose $a,b\in\mathbb{Z}$.  If $4\mid (a^2 + b^2)$, then
  $a$ and $b$ are not both odd.
  
  Proposition: Suppose $a,b\in\mathbb{Z}$.  If $4\mid (a^2 + b^2)$, then
  $a$ and $b$ are not both odd. (Can be even-even or odd-even but not odd-odd)
  
  \textit{Proof.} Suppose that there exist values of $a$ and $b$ that are both odd, but still allow $4\mid(a^2+b^2)$ to hold true.
  
  By definition of an odd number, both $a$ and $b$ must exist within the set of numbers: $\{2n+1:n\in\mathbb{Z}\}$
  
  Further, in being divisible by 4, $(a^2+b^2)$ must sum to some number $c$ that lies within the set defined as $\{4n:n\in\mathbb{Z}\}$
  
\item[Exercises for Chapter 7]
  State clearly which method of proof you are using.
\item[24.]  If $a\in\mathbb{Z}$, then $4\nmid (a^2-3)$.

Proposition: if $a\in\mathbb{Z}$, then $4\nmid (a^2-3)$

\textit{Direct proof.} As stated in problem 12 under Exercise 5 above, if $a^2$ is not divisible by 4,
then $a$ must be odd. It follows then, that $a$ must be even if $a^2$ is divisible by 4.

For all even values of $a$, $4\nmid(a^2-3)$ must hold true. An even value of $a$ would produce an even valued $a^2$
as justified by: $(2n)*(2n) = (4n^2)$, which, for all values of n still produces an even number. 


  

\item[Exercises for Chapter 8]
\item[20.] Prove that $\{9^n:n\in\mathbb{Q}\}= \{3^n:n\in\mathbb{Q}\}$.  

\item[Exercises for Chapter 9]
  Each of the following statements is either true or false.
  If a statement is true, prove it.
  If a statement is false, disprove it.
\item[18.] If $a,b,c\in\mathbb{N}$, then at least one of $a-b$,
  $a+c$, and $b-c$ is even.

  

\item[Exercises for Chapter 10]
  
\item[2.] For every integer $n\in\mathbb{N}$, it follows that
  \[
  \sum_{i=1}^n i^2 = \frac{n(n+1)(2n+1)}{6}
  \]

\item[6.] For every natural number $n$, it follows that
  \[
  \sum_{i=1}^{n}(8i-5) = 4n^2 - n
  \]

\item[10.] For every integer $n\geq 0$, it follows that $3\mid (5^{2n} - 1)$.


\item[14.]
 Suppose $a\in\mathbb{Z}$.  Prove that $5\mid 2^na$ implies
  $5\mid a$ for any $n\in\mathbb{N}$.

\end{description}


\end{document}