\documentclass{article}
\usepackage{amsfonts,amssymb}
\usepackage[margin=1in]{geometry}

\title{CSCI 301, Winter 2017\\Math Exercises \# 3}
\author{YOUR NAME HERE}
\date{Due date:  Friday, February 10, midnight.}

\begin{document}

\maketitle

\begin{description}
\item[Exercises for Section 11.1]

\item[12.] Prove that the relation $\mid$ (divides) on the set
  $\mathbb{Z}$ is reflexive and transitive.
  

\item[Exercises for Section 11.2]
\item[8.] Define a relation $R$ on $\mathbb{Z}$ as $xRy$ if and only if
  $x^2 + y^2$ is even.  Prove $R$ is an equivalence relation (prove
  from the definitions).  
  
\item[Exercises for Section 11.4]
\item[4.] Write the addition and multiplication tables for $\mathbb{Z}_6$

\item[Exercises for Section 12.2]
\item[8.]  A function $\mathbb{Z}\times\mathbb{Z} \rightarrow
  \mathbb{Z}\times\mathbb{Z}$ is defined as $f(m,n) = (m+n, 2m+n)$.
  Verify whether this function is injective and whether it is
  surjective.

  
\item[Exercises for Section 12.3]
\item[2.]  Prove that if  $a$ is a natural number,
  then there exist two unequal natural numbers $k$ and $\ell$
  for which $a^k - a^\ell$ is divisible by 10.


\item[Exercises for Section 12.5]
\item[6.]   The function
  $\mathbb{Z}\times\mathbb{Z} \rightarrow \mathbb{Z}\times\mathbb{Z}$
  defined by the formula
  $f(m,n) = (5m+4n,4m+3n)$
  is bijective.  Find its inverse.
  

\item[Exercises for Section 12.6]
\item[6.]  Given a function $f:A\rightarrow B$ and a subset
  $Y\subseteq B$, is $f(f^{-1}(Y)) = Y$ always true?  Prove or
  give a counterexample.

\item[Exercises for Section 13.1]
\item[A.] Show that the two given sets have equal cardinality
  by describing a bijection from one to the other.  Describe
  your bijection with a formula (not as a table).
\item[10.] $\{0,1\}\times\mathbb{N}$ and $\mathbb{Z}$


\item[Exercises for Section 13.2]
\item[14.]  Suppose $A=\{(m,n)\in\mathbb{N}\times\mathbb{R}: n = \pi m\}$.
  Is it true that $|\mathbb{N}| = |A|$?  Prove or disprove it.


\item[Exercises for Section 13.3]
\item[8.] Prove or disprove:  The set
  $\{(a_1,a_2,a_3,\ldots):a_i\in\mathbb{Z}\}$ of
  infinite sequences of integers is countably infinite.
  

\end{description}


\end{document}