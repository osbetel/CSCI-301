\documentclass[10pt]{article}
\usepackage[usenames]{color} %used for font color
\usepackage{amssymb} %maths
\usepackage{amsmath} %maths
\usepackage[utf8]{inputenc} %useful to type directly diacritic characters

\title{CSCI 301, Winter 2017\\Math Exercises \# 1}
\author{Andrew Nguyen}
\date{Due date:  Monday, January 23, midnight.}

\begin{document}



\maketitle

\noindent{\large
  {\bf Turn in both the tex and pdf files (not zipped):  } {\tt math01.tex} and {\tt math01.pdf}.  }

\begin{description}
\item[Exercises for Section 1.1]
\item[C.] Find the following cardinalities
\item[30.] $|\{\{1,4\},a,b,\{\{3,4\}\},\{\emptyset\}\}|$ = 5

\item[Exercises for Section 1.3]
\item[A.] List all the subsets of the following sets.
\item[8.] $\{\{0,1\},\{0,1,\{2\}\},\{0\}\}$


Let: 
a = $\{0, 1\}$ 

b = $\{0,1,\{2\}\}$ 

c = $\{0\}$

If O = the original set, then $\{a,b,c\}$ = O
All subsets of O would be $\{a,b,c\}$, $\{a,b\}$, $\{a,c\}$, $\{b,c\}$, $\{a\}$, $\{b\}$, $\{c\}$, $\{\emptyset\}$


\item[Exercises for Section 1.4]
\item[A.] Find the indicated sets.
\item[12.] $\{X\in\mathcal{P}(\{1,2,3\}):2\in X\}$

All sets where X is an element of $\mathcal{P}(\{1,2,3\})$ such that 2 is an element of X.

$\mathcal{P}(\{1,2,3\}) = \{\{ø\}, \{1\}, \{2\}, \{3\}, \{1, 2\}, \{1, 3\}, \{2, 3\}, \{1, 2, 3\}\}$

Answer = $\{\{2\}, \{1, 2\}, \{2, 3\}, \{1, 2, 3\}\}$ 

\item[B.] Suppose that $|A|=m$ and $|B|=n$.  Find the following cardinalities.
\item[18.] $|\mathcal{P}(A\times \mathcal{P}(B))|$

$|\mathcal{P}(B))| = 2^n$

$|(A\times 2^n)| = (m2^n)$

Thus, $|\mathcal{P}(A\times \mathcal{P}(B))| = 2^{m2^n}$

\item[Exercises for Section 2.5] Write a truth table for the logical
  statements in problems 1--9:

\item[8.] $P \vee (Q \wedge \neg R)$

 \begin{tabular}{||c c c c||} 
 \hline
 P & Q & R & Result \\ [0.5ex] 
 \hline\hline
 T & T & T & T\\ 
 \hline
 T & T & F & T\\
 \hline
 T & F & T & T\\
 \hline
 T & F & F & T \\
 \hline
 F & T & T & F \\
 \hline
 F & T & F & T \\
 \hline
 F & F & T & F \\
 \hline
 F & F & F & F \\ [1ex] 
 \hline
\end{tabular}


\item[Exercises for Section 2.10]  Negate the following sentences.


\item[8.] If $x$ is a rational number and $x\not = 0$, then
  $\tan(x)$ is not a rational number.

If $x$ is an irrational number or $x = 0$, then
$\tan(x)$ is a rational number.
  

\item[Exercises for Section 3.1]

\item[4.]  Five cards are dealt off a standard 52-card deck and
  lined up in a row.  How many such lineups are there in
  which all 5 cards are of the same suit?

I am assuming due to the use of the word "lineups" that the order you draw the cards in is significant. \\\\
i.e.: (King, Ace, 1, 2, 3) is different than (Ace, King, 1, 2, 3) 

$_nP_r(13, 5) = 154,440$ different ways. \\ 
$(13 * 12 * 11 * 10 * 9) = 154,440$ too. \\\\
It makes more sense to think of it that way since we have 13 choices at first, 
select 1, 12 are left, select one more, etc. This is the calculation assuming we 
draw only cards of one suit, say specifically diamonds. So we multiply this 
by 4 because it applies to each suit in the deck and we get the total ways 
to get 5 cards of the same suit.: \\ $4 * 154,440 = 6,177,600$ \\


\item[Exercises for Section 3.2]

\item[8.] Compute how many 7-digit numbers can be made from the digits
  1,2,3,4,5,6,7 if there is no repetition and the odd digits
  must appear in an unbroken sequence.  (Examples: 3571264 or 2413576 or
  2467531, but not 7234615.)

Since the odd numbers can’t break sequence, they are limited to “slots” of four spaces. 
The odd numbers must go in the first four, second four, third four, or final four slots, 
leaving the even numbers to take up the rest of the space. 

For the first four spaces, the odd numbers can be arranged in 4! ways, and the remaining 
even numbers have $3!$ orientations giving $4!3! = 144$ ways only for the case of odd numbers 
in the first four slots. The same applies to the cases of the second, third, and final four 
space slots. So $144 * 4 = 576$ total different scenarios.

\item[Exercises for Section 3.3]

\item[12.] Twenty-one people are to be divided into two teams, the Red
  Team and the Blue Team.  There will be 10 people on the Red Team and
  11 people on the Blue Team.  In how many ways can this be done?

The answer is $_nC_r(21, 10) = 116,280$

\item[Exercises for Section 3.5]


\item[8.] This problem concerns 4-card hands dealt off of a standard
  52-card deck.  How many 4-card hands are there for which all 4 cards
  are of different suits or all 4 cards are red?


The total number of 4-card hands is the sum of both the first and second scenarios.

If all cards are different suits, you would be choosing 1 card from 13 (13 per suit), 
4 times in a row. Or simply $13^4 = 28,561$ ways to choose 4 cards that all have different suits.

If you wanted all 4 cards to be red, then you are choosing 4 cards out of 26 red cards. $_nC_r(26, 4) = 14,950$ ways to choose 4 red cards. 
This is all assuming the order of the cards does not matter (I’m assuming it doesn’t since it wasn’t specified in the question).
The total number of hands that satisfy one of the two conditions above is $28,561 + 14,950 = 43,511$



\end{description}


\end{document}